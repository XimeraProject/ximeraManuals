\documentclass{ximera}

\title{GitHub setup}

\author{Bart Snapp}

\begin{document}
\begin{abstract}
    How to create a new Ximera project.
\end{abstract}
\maketitle

\paragraph{All Ximera files must be hosted in a git repository.}
As an author, you have some choices as to how you proceed.

\paragraph{Starting fresh} is easy.
We have two repositories potential authors can use as a template:
\begin{itemize}
    \item \url{https://github/ximeraProject/ximeraFirstSteps} for a template
          with multiple files that you can edit.
    \item \url{https://github/ximeraProject/ximeraNewProject} for a more
          minimal template.
\end{itemize}
If you go to either of these repositories when you are logged in to GitHub, you
will have the option to make a new repository using the green ``Use this
template'' button. Once the template is made, you are ready to go!

\paragraph{Deploy an existing repository from a codespace} requires some
extra files. Copy the
following files from either \verb!ximeraFirstSteps! or \verb!ximeraNewProject!
into the repository you wish to deploy.
\begin{description}
    \item \texttt{.devcontainer/} (a folder you need if you want to deploy from
          a
          GitHub codespace)
    \item[\texttt{xmScripts/}] This is a folder that contains our Ximera
        scripts.
    \item[\texttt{.vscode/}] This is a folder that contains our VS Code
        settings, including the buttons that allow Ximera compilation.
    \item[\texttt{.xmKeyFile}] This is a `dummy' gpg-key file. If you want to
        deploy from a GitHub codespace and preview your work, you need this
        file.
    \item[\texttt{.gitignore}]	If you already have a \verb|.gitignore| file,
        we suggest
        you \textbf{replace yours with ours.}
\end{description}
Commit your changes to GitHub and view your repository online via a web browser
to ensure that the files were added.
You can further check your work by starting a codespace in your repository.

\paragraph{Deploy an existing repository from your machine} is for experienced
users who can install software on their own computer. In particular, you will need to install:
\begin{description}
    \item[Docker] is a development utility allow you to easily run self-contained applications on your
computer. You must install and start Docker before you can deploy. You can just let
it run in the background. 
\item[VS Code Desktop] is a popular text-editor. It is has strong \LaTeX, git,
and Docker integration. It provides a UNIX-like command line interface on
Windows machines via Microsoft's WSL (Windows Subsystem for Linux).
\end{description}
Once these applications are installed, a typical workflow will be:
\begin{enumerate}
    \item Open Docker and minimize the window.
    \item Open Visual Studio Code, do File $\to$ Open Folder, and select the
          folder of your git repository.
    \item To open files, do \verb!Ctrl-p! and start typing file names. Any
          file
          committed to your git repository will be found, and files in
          \verb!.gitignore!, will not be shown.
    \item To run a special command (like search and replace) do
          \verb!Ctrl-Shift-p!, and search for the command.
    \item To toggle a UNIX-like terminal, use \verb!Ctrl-~!.
\end{enumerate}

Now, copy the following files from either \verb!ximeraFirstSteps! or \verb!ximeraNewProject!
into the repository you wish to deploy.
\begin{description}
    \item[\texttt{.devcontainer/}] (only needed if you want to additionally deploy from
          a GitHub codespace)
    \item[\texttt{xmScripts/}] This is a folder that contains our Ximera
        scripts.
    \item[\texttt{.vscode/}] This is a folder that contains our VS Code
        settings, including the buttons that allow Ximera compilation.
    \item[\texttt{.gitignore}]	If you already have a \verb|.gitignore| file,
        we suggest
        you \textbf{replace yours with ours.}
\end{description}
Commit your changes to GitHub and view your repository online via a web browser
to ensure that the files were added. Now, move the \verb!.xmKeyFile! to your repository.
\textbf{If you are deploying from your machine, do not commit the \texttt{.xmKeyFile} to the repository.}








\end{document}
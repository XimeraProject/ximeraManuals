\documentclass{ximera}

\title{Create an activity}

\author{Bart Snapp}

\begin{document}
\begin{abstract}
    How to set up an activity.
\end{abstract}
\maketitle

Ximera content consists of two distinct types of \LaTeX\ files. Those with the
\verb!xourse! document class and those with the \verb!ximera! document class.
Files with the \verb!xourse! document class \textit{glue} together Ximera content via
the commands \verb!\activity! and \verb!\practice!. Ximera content is written
using the  \verb!ximera! document class.
The basic structure of a Ximera activity is as follows:
\begin{verbatim}
\documentclass{ximera}
\begin{document}
%%
%% Content goes here
%%
\end{document}
\end{verbatim}

A Ximera activity may be a single problem, a list of exercises, a review of a
concept, a section of a book---you are only limited by your imagination.
Examples of individual Ximera activities can be found here:
\begin{description}
    \item[Single problem]See:

    \url{https://ximera.osu.edu/mooculus/limitLaws/exercises/exerciseList/limitLaws/exercises/limitLaw2}
    \item[List of exercises] See:

        \url{https://ximera.osu.edu/mooculus/limitLaws/exercises/exerciseList/limitLaws/exercises/limitLaw7}
    \item[Review of concept] See:

        \url{https://xronos.clas.ufl.edu/mac1140nowell/PrecalculusXourse/explorePolynomials/Practice/factoringGeneral-Practice1}
    \item[Section of book] See:

        \url{https://ximera.osu.edu/mooculus/calculus1/limitLaws/digInLimitLaws}
\end{description}

You can see the source code for any of the activities above by appending
\verb!.tex! to the URL.

\section{Tips for Ximera activities}

We wish to write Ximera documents as simply as possible. Here are some tips.

\paragraph{Give them descriptive file names} so that someone can figure out what file they are looking at. File names like \verb!problem1.tex! are usually less helpful than say, \verb!basicDerivativePractice1.tex!.
In particular, all document  names used for Ximera must be web-safe! This means
    they:
    \begin{description}
        \item[Must end with \texttt{.tex}] as our compiler will not attempt to compile non-\TeX\ documents.   
      \item[Must only use alphanumeric English characters] meaning: a,b, \dots,
        z, A,B, \dots, Z, 0,1, \dots, 9, and hyphen `-' and underscore `\_'
        though the
        last two are discouraged.
      \item[Cannot use any other characters, including spaces] meaning all
        Ximera document names must be a single word.
    \end{description}
    This is not a limitation of Ximera, rather it is a rule that nearly all
    web-accessible documents must follow.


\paragraph{Use basic \LaTeX} environments and commands like:
\verb!enumerate!,
\verb!itemize!,
\verb!description!,
\verb!align*!, \verb!$...$!, \verb!\[...\]!, and so on.
Macros made of basic \LaTeX\ commands are fine as long as you store them in
\verb!xmPreamble.tex! or another file on your input path. This helps keep the headers very clean.


\paragraph{Avoid manual formatting.} This means, that you shouldn't try to vertically space your content and you should use \LaTeX\ commands for semantic markup.
At the core of the Ximera philosophy is
the principle of separating content from deployment. This means that the
intellectual substance---the ideas, explanations, and activities---should
remain independent of the specific presentation or layout in which they are
delivered. This separation ensures adaptability, longevity, and accessibility, enabling
content to evolve without being constrained by the limitations of any
particular delivery method.


With that said, our strategies for creating PDFs are to load a separate style
file at the level of a \verb!xourse! file. Further formatting can be achieved via the command \verb!\pdfOnly!.

\paragraph{Special Ximera documents and folders} are provided and are \textbf{automatically}
loaded by all files in the repository (assuming they are no more than three
folders deep).
\begin{description}
    \item[\texttt{xmPreamble.tex}] This is a place to put custom commands to
        help your document compile. It is not intended to be used for cosmetic
        changes
        to Ximera. For making pretty documents, we load a separate preamble (usually called a print style) protected by the command \verb!\pdfOnly!.
    \item[\texttt{xmPictures/}] This is a folder that you can place JPGs,
        PNGs, and PDFs, for inclusion into Ximera documents.
\end{description}



\paragraph{If you use \texttt{\textbackslash maketitle},} please ensure you have a title and an abstract. Something like this will suffice:
\begin{verbatim}
\documentclass{ximera}
\title{My groovy activity}
\begin{document}
\begin{abstract}
    A really fun math activity!
\end{abstract}
\maketitle

\end{document}
\end{verbatim}




\end{document}

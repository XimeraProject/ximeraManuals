\documentclass{ximera}

\title{Preambles, input paths, and graphics paths}

\author{Bart Snapp}

\begin{document}
\begin{abstract}
  How to add extra packages and commands.
\end{abstract}
\maketitle

The Ximera documentclass comes with many packages preloaded:
\verb!enumitem!,
\verb!titlesec!,
\verb!titletoc!,
\verb!titling!,
\verb!url!,
\verb!xcolor!,
\verb!tikz!,
\verb!pgfplots!,
\verb!fancyvrb!,
\verb!forloop!,
\verb!environ!,
\verb!amssymb!,
\verb!amsmath!,
\verb!amsthm!,
\verb!xifthen!,
\verb!multido!,
\verb!listings!,
\verb!xkeyval!,
\verb!comment!,
\verb!gettitlestring!,
\verb!nameref!,
\verb!epstopdf!,
\verb!hyperref!.

The documentclass also provides support for the following theorem-like
environments:
\verb!algorithm!, \verb!axiom!, \verb!claim!, \verb!conclusion!,
\verb!condition!, \verb!conjecture!, \verb!corollary!, \verb!criterion!,
\verb!definition!, \verb!example!, \verb!explanation!, \verb!exercise!,
\verb!exploration!,
\verb!fact!, \verb!formula!, \verb!hypothesis!, \verb!idea!, \verb!lemma!,
\verb!model!,
\verb!notation!, \verb!observation!, \verb!paradox!, \verb!proof!,
\verb!problem!,
\verb!procedure!,
\verb!proposition!, \verb!question!, \verb!remark!, \verb!solution!,
\verb!summary!, \verb!template!, \verb!theorem!, \verb!warning!.

We can add functionality and generality, via user defined commands, to the
document via a preamble, or macros, document. Currently, we have a special file
called \verb!xmPreamble.tex! that we strongly suggest new users use.
The purpose of \verb!xmPreamble! is to ensure \textbf{all files compile
  consistently,
  it is not for cosmetic changes.} A typical preamble might contain things
like:

\begin{verbatim}
\newcommand{\R}{\mathbb{R}}
\newcommand{\d}{\, d}
\end{verbatim}

\begin{warning}
  Cosmetic changes to Ximera \textbf{environments} will result in unpredictable
  behavior
  online.
\end{warning}

\paragraph{The special file \texttt{xmPreamble.tex}} is a preamble (macros file) that is \textbf{automatically loaded} by every  \verb!ximera! and \verb!xourse! file.
Since every \verb!*.tex! file with a \verb!ximera! and \verb!xourse! must compile, the
preamble file must be accessible  by all files. The special file
\verb!xmPreamble.tex! is found by all files that are at most three directories
deep. In essence, we are modifying the input path as follows:
\begin{verbatim}
\makeatletter      %% make "@" a letter-character
\def\input@path{   %% When looking for xmPreamble.tex,
{./}               %% look here first,
{../}              %% then look a folder up,
{../../}           %% then look two folders up,
{../../../}        %% then look three folders up.
}
\makeatother       %% make "@" an other-character.
\end{verbatim}
and we are \textbf{automatically} loading any file it finds that is named \verb!xmPreamble.tex!.



\paragraph{The special folder \texttt{xmPictures/}} helps \verb!ximera! and  \verb!\xourse! files find images.
We used \verb!\includegraphics! in \texttt{basicWorkSheet.tex} with code like:
\begin{verbatim} 
\begin{center}
  \includegraphics[width=5cm]{missionPatch.jpg}
\end{center}
\end{verbatim}
In this case, \texttt{aFirstActivity.tex}  and \verb!missionPatch.jpg! are both
in the folder
\texttt{basics}. However, when
this document is compiled via a \verb!xourse! document, the \LaTeX\ compiler
looks for the image \textit{at the level of the
  \texttt{aFirstStepInXimera.tex}}.

Since there is no file \verb!missionPatch.jpg! at the
level of the
\verb!xourse! document the compilation will fail unless the files know where to look. To solve this
problem we use the document class to append the graphics path. Essentially we set:
\begin{verbatim}
  \graphicspath{          %% When looking for images,
  {./}                    %% look here first,
  {.\xmPictures}          %% then look in \xmPictures,
  {..\xmPictures}         %% then look a folder up,
  {../..\xmPictures}      %% then look two folders up,
  {../../..\xmPictures}   %% then look three folders up.
  }
  \end{verbatim}
With this code, Ximera files can find images. 



\paragraph{Input paths} allow you to load other documents into your files.
This is a less common use case for power-users, but we'll mention it here. You can surely avoid
the use of such files with an enhanced preamble. However, for this document, we
needed to input various TikZ files to help us draw our diagrams. Hence we
needed to
add the location of these files to our input path. We did this in our \verb!xmPreamble.tex!
the following way:
\begin{verbatim}
%% Where to look for inputs
\makeatletter     %% make "@" a letter-character
\def\input@path{  %% When looking for files,
{./}              %% look first at your level
{./xmPictures/}   %% then in this folder,
}
\makeatother      %% make "@" an other-character.
\end{verbatim}
With this said, modifying input paths is more of an advanced feature and one
should be careful.
\end{document}

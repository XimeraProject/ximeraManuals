\documentclass{ximera}

\title{Common issues}
\author{Bart Snapp}

\begin{document}
\begin{abstract}
    Suggestions for dealing with common issues.
\end{abstract}
\maketitle

\section{Compilation issues}

While we seek to support as much of \LaTeX\ as possible, there are features
that are either not implemented, difficult to implement, or impossible to
implement.

\paragraph{If the SERVE fails} and you are working in a codespace, then you can
use \verb|xake| directly to debug your code.
You can use \verb!pdflatex! to compile and use
\begin{verbatim}
pdflatex FILE-NAME.tex
xake -v compile FILE-NAME.tex
more FILE-NAME.tex.log
\end{verbatim}
If  you find a file that does not compile, you can try to compile the file
directly while
slowly moving \verb!\end{document}! down through the document. By starting at
the top, the file should compile, and then you should be able to locate the
exact location of the bug.

\paragraph{If a \LaTeX\ package is unsupported} and you need to use it, you can
modify \verb!xmScripts/config.txt!
\begin{verbatim}
 # version of Ximera xake container to use (with xmlatex)
 #
 # : "${XAKE_VERSION:=v2.4.2-full}"   # with texlive_full
\end{verbatim}
by removing the \verb!#! in front of the container ending in ``full.'' Note,
this is a \textbf{large} container, and hence is not shipped by default.

\paragraph{If codespace is not acting correctly,} you can always ``Delete'' it,
and make a new one.

\paragraph{Files in \texttt{xmScripts/}} must be executable. If there is an issue with
permissions, open the codespace and run
\begin{verbatim}
chmod +x ~/xmScripts/xmlatex*
\end{verbatim}
In a codespace (or inside a Docker container), \verb!pdflatex! and \verb!xake!
(Ximera's make program)  are available and the \verb!xmlatex! script is in your
PATH. This means that the following commands are available to the user:
\begin{verbatim}
xmlatex compilePdf <path-to-your-texfile>
xmlatex compile <path-to-your-texfile>
xmlatex bake
xmlatex serve
xmlatex updateDevEnv
\end{verbatim}
The final command will update your authoring environment to our most recent
release.

\section{Online issues}

Sometimes there is strange behavior online. If this happens, start by
\textbf{visiting the offending URL in a private browser}. If the content looks
good in a private browser, than the issue is due to cookies or browser
settings.

\paragraph{An svg-viewer missing} error message means that there was a
compilation error that confused Xake. The offending file needs to be
recompiled. Make a trivial change in the file, and delete all SVG files in the
directory.

Sometimes Ximera materials look good to instructors and others, but not to some
individual students.

\paragraph{Ximera refreshes the page when submitting}
can be caused by several issues, each with a different solution:
\begin{description}
    \item[Problem:] Poor internet connection
        \begin{description}
            \item[Solution:] Refresh the page immediately before entering each
                answer.
        \end{description}
    \item[Problem:] Browser settings
        \begin{description}
            \item[Solution:] Try using a different browser. Different students
                have
                reported luck with different browsers, including Edge, Firefox,
                and Chrome.
        \end{description}
    \item[Problem:] Antivirus Software
        \begin{description}
            \item[Solution:] Some antivirus software (such as McAfee) are now
                blocking
                websockets, a technology Ximera relies on. Try turning off the
                websockets
                setting on the antivirus software.

        \end{description}
    \item[Problem:] VPN
        \begin{description}
            \item[Solution:] If you are using a VPN, try turning it off before
                using
                Ximera.
        \end{description}
\end{description}

\paragraph{unknown node type: parser error} message. This sometimes appears
rather than the actual rendered mathematics. This is due to
a
corruption in the cookies/cache for the browser, and the person
seeing this
error need to clear their cache and then reload the page to resolve
the
problem.

\paragraph{Difficulty accessing Ximera} is often caused by browser settings.
The
easiest solution is usually to try using a different browser. Students seem to
have the best luck with Chrome, Firefox, or Safari.

\paragraph{Math processing error}

This is caused by an error in the browser setting. Two things to try:

First possible solution: Clear browser cookies. Close the browser.
Open the browser and log in to the LMS (if applicable). Go to the assignment.

Second possible solution:
Find one of the pages this is happening,
Put your mouse over the `math processing error'
Now, `right-click' and select `accessibility' If collapsible math is
checked, uncheck it. Reload the page.

%% Exclusively for LMS
% \paragraph{Using Ximera on an iPad}

% Be sure to access Ximera via the LMS using a browser such as Safari or Chrome.
% If one is having trouble accessing the Math Editor, be sure you have
% updated your iPad software to the latest version.
% If you are having trouble with auto-capitalization in Ximera, you can:
% Hit shift twice before entering an answer to get a lower case letter.
% Turn off auto-capitalization on your iPad by going to Settings $\to$ General
% $\to$ Keyboards and toggling off auto-capitalization.

\end{document}

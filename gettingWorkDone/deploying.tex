\documentclass{ximera}

\title{Deploying Ximera documents}
\author{Bart Snapp}

\begin{document}
\begin{abstract}
    Deploying Ximera documents.
\end{abstract}
\maketitle

We have two options for deploying Ximera documents: Deploying through a GitHub
codespace and deploying from your machine.

\section{Deploying through codespace}

Assuming your repository contains the folder \verb!.devcontainer! from either
\verb!ximeraFirstSteps! or \verb!ximeraNewProject!, you (actually anyone!) can
start a codespace from GitHub in your web browser. We've discussed how to
deploy to a temporary  server running in the cloud. To deploy to public-visible
Ximera server such as:
\begin{center}
    \url{https://ximera.osu.edu}
\end{center}
you will need to do some additional set up. 

\section{Deploying from your machine}

\begin{enumerate}
    \item Make sure all source files are committed and pushed to the
          repository. A quick
          \begin{verbatim}
git add -u && git commit -m "this is my change" && git push
\end{verbatim}
          may help. Also, you can check your personal GitHub page to ensure
          files are in
          the repository.
    \item Ensure Docker is running.
    \item Press \textit{Bake} (typically on the bottom taskbar in VS Code).
    \item Press \textit{Serve} (typically on the bottom taskbar in VS Code).
\end{enumerate}

We should discuss this more once the discussion in the repository
\verb!ximeraFirstSteps! is steady.

\end{document}

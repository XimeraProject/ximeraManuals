\documentclass{ximera}

\author{Bart Snapp and Jeff Kuan}

\title{Accessibility}

\begin{document}
\begin{abstract}
Ximera strives to be accessible for all users.
\end{abstract}
\maketitle

\section{Introduction}
Ximera strives to be compliant with the Web Content Accessibility Guidelines (WCAG) 2.1 AA standard, as required by the United States
Department of Education. These are the technical standards that will be required of public colleges and universities in April 2026, 
according to the Department of Justice's interpretation of Title II of the Americans with Disabilities Act. 
We are currently partnering with \link[Tailor Swift Bot]{https://tailorswiftbot.com/} 
to ensure that our content meets these standards.

Of course, meeting a technical standard does not imply that content will still be accessible to every user with a disability. 
It is important that instructors include an \link[accessibility statement]{https://www.w3.org/WAI/planning/statements/} for their
students. Depending on the institution's organization, the contact person may be the disability resources office, an instructional designer, 
or even another faculty member. If you wish to verify that your course meets WCAG2.1AA, you may email \link[Jeffrey Kuan]{mailto:kuan.44@osu.edu},
who can use a variety of screen readers and automatic accessibility checkers. Do NOT claim that your course is WCAG2.1AA compliant
without a manual human verification. 

\section{Current features}

\subsection{MathJax}
Webpages created with Ximera will have the mathematical content displayed in MathJax, which can be read with EquatIO. In principle, the webpage 
can be edited so that screenreaders like NVDA, JAWS, VoiceOver and Narrator can read it directly. We would recommend checking if your
institution licenses EquatIO for your students. 

\subsection{Lists}
Using itemize, enumerate and description environments in LaTeX will correctly create unordered, ordered and description lists (respectively) in HTML. We do not offer custom list labeling, although
labels can be modified in the CSS file (this will be done later). The references list will be displayed as a description list, which is the \link[recommendation of the DAISY consortium]{http://kb.daisy.org/publishing/docs/html/bibliographies.html}.

\subsection{Desmos and Geogebra}
Both \link[Desmos]{https://www.desmos.com/accessibility} and \link[Geogebra]{https://help.geogebra.org/hc/en-us/articles/20048444963869-Accessibility} have accessibility features.


\section{Upcoming features}

\subsection{MathML4.0}
As of the current date (June 25, 2025), we are still waiting for the final release of MathML4.0, which is planned for August 2025.
One notable feature will be the ``intent'' environment, which allows the author to insert their intended text--to--speech for
math symbols. For example, 
\[
a(b+c) = ab + ac
\]
can be read as ``a times the quantity b plus c equals a b plus a c'' rather than ``a left paren b plus c right paren equals ab plus ac.''
Historically, this would be called ``alternative text''; for a variety of reasons, that terminology is somewhat outdated. 
We will add a LaTeX command or environment, probably called ``intent'', that will insert the author's intended text--to--speech in the
webpage. 

\subsection{Tables}
Tables will need to have headers and scopes from the LaTeX code. This will soon be implemented with tabularray. 


\subsection{Headers}
The problem, exercise, and question environments will need to have appropriate headers; same with theorem, proposition, etc. 


\subsection{Answer}
In the future, screen readers will be able to inform readers if their answers are correct or incorrect.  

\subsection{Navigation}
We will work with experts at Ohio State University who have worked on the navgiability of webpages. 





\end{document}
\documentclass{ximera}

\title{Printing with style}

\author{Bart Snapp}

\begin{document}
\begin{abstract}
    We describe methods of styling the print versions of Ximera materials.
\end{abstract}
\maketitle

\section{Documentclass options}

There are a number of options for the document class, though their
effects are only seen in the PDF:

%% Copied from dtx
\begin{description}
    \item[\tt\bfseries handout] The default behavior of the class is to display
        \textbf{all} content. This means that if any questions are asked, all
        answers
        are shown. Moreover, some content will only have a meaningful
        presentation when
        displayed online. When compiled without any options, this content will
        be shown
        too. This option will supress such content and generate a reasonable
        printiable
        ``handout.''
    \item[\tt\bfseries noauthor] By default, authors are listed at the bottom
        of
        the first page of a document. This option will supress the listing of
        the
        authors.
    \item[\tt\bfseries nooutcomes] By default, learning outcomes are listed at
        the
        bottom of the first page of a document. This option will supress the
        listing of
        the learning outcomes.
    \item[\tt\bfseries instructornotes] This option will turn on (and off)
        notes
        written for the instructor.
    \item[\tt\bfseries noinstructornotes] This option will turn off (and on)
        notes
        written for the instructor.
    \item[\tt\bfseries hints] When the \texttt{handout} option is used, hints
        are
        not shown. This option will make hints visible in handout mode.
    \item[\tt\bfseries newpage] This option will start each problem-like
        environment (\texttt{exercise}, \texttt{question}, \texttt{problem},
        and
        \texttt{exploration}) on a new page.
    \item[\tt\bfseries numbers] This option will number the titles of the
        activity.
        By default the activities are unnumbered.
    \item[\tt\bfseries wordchoicegiven] This option will replace the choices
        shown
        by \texttt{wordChoice} with the correct choice. No indication of the
        \texttt{wordChoice} environment will be shown.
\end{description}

\section{The preamble versus printing styles}

\paragraph{The preamble} is used to make the Ximera document compile. You include things like:
\begin{verbatim}
\newcommand{\R}{\mathbb R}
\renecommand{\vec}{\mathbf}
\end{verbatim}


\paragraph{The printing style} is used to make cosmetic changes to the PDF. They are only
included in the \verb!xourse! files and are enclosed in \verb!\pdfOnly!. For
example the printstyles for this document are loaded with
\begin{verbatim}
\pdfOnly{\usepackage{manual}}
\end{verbatim}
In the print styles, you can change how the document looks cosmetically with things like:
\begin{verbatim}
\let\warning\relax
\let\endwarning\relax
\newtheoremstyle{warning}
{\topsep}{\topsep}{\rmfamily}{}{\bfseries\sffamily}{:}{ }{#1}
\theoremstyle{warning}
\newtheorem*{warning}{WARNING}
\surroundwithmdframed[innertopmargin=10pt]{warning}
\end{verbatim}


% \section{Online versus PDF}

% The Ximera document class provides special commands and environments to
% distinguish between content that is shown online and content that is shown in
% the PDF. THe commands are
% \begin{verbatim}
% \pdfOnly{CONTENT THAT IS ONLY SHOWN IN THE PDF}
% \htmlOnly{CONTENT THAT IS ONLY SHOWN ONLINE}
% \end{verbatim}
% And the environments are
% \begin{verbatim}
% \begin{onlyHtml}
% CONTENT THAT IS ONLY SHOWN ONLINE
% \end{onlyHtml}

% \begin{onlyPDF}
% CONTENT THAT IS ONLY SHOWN IN THE PDF
% \end{onlyPDF}
% \end{verbatim}

% \begin{onlyPDF}
%   CONTENT THAT IS ONLY SHOWN IN THE PDF
%   \end{onlyPDF}

\section{When and how to use \texttt{prompt}}

Prompt allows you to make your exercises look beautiful in the PDF.  As an
example consider this exercise:
\begin{verbatim}
\begin{exercise}
Compute $\frac{d}{dx} x^2$
\begin{prompt}
\[
  \frac{d}{dx} x^2 = \answer{2x}
\]
\end{prompt}
\end{exercise}
\end{verbatim}
It renders as 
\begin{exercise}
Compute $\frac{d}{dx} x^2$
\begin{prompt}
\[
  \frac{d}{dx} x^2 = \answer{2x}
\]
\end{prompt}
\end{exercise} 
The environment \verb!prompt! hides the display math in the PDF. In a PDF, you don't need answer boxes!
As another example, consider:
\begin{verbatim}
\begin{exercise}
 Consider the planes defined by the equations below.
\begin{align*}
  P_1:  \quad 4 &= 2x-y+3z  \\
  P_2:  \quad 5 &= 4x-2y+6z \\ 
  P_3:  \quad 7 &= 5x+2y+z
\end{align*}
Describe the relationships between the planes 
$P_1$, $P_2$, and $P_3$ in terms of ``parallel,'' 
``orthogonal,'' or ``neither.''
\begin{prompt}
\begin{enumerate}
  \item The planes $P_1$ and $P_2$ are \wordChoice{
    \choice[correct]{parallel}
    \choice{orthogonal}
    \choice{neither parallel nor orthogonal}
    }
  \item The planes $P_1$ and $P_3$ are \wordChoice{
    \choice{parallel}
    \choice{orthogonal}
    \choice[correct]
    {neither parallel nor orthogonal}}.
  \item The planes $P_2$ and $P_3$ are \wordChoice{
    \choice{parallel}
    \choice{orthogonal}
    \choice[correct]{neither parallel nor orthogonal}
    }
\end{enumerate}
\end{prompt}
\end{exercise}
\end{verbatim}
The environment \verb!prompt! says that the stuff inside the environment is
used as a \textit{prompt} for the student. In the PDF, everything within the
prompt is hidden, as a student doesn't need that content to answer the question
using pencil and paper.

\end{document}

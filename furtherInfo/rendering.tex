\documentclass{ximera}
\title{Rendering content}

\author{Jason Nowell \and Bart Snapp}
\begin{document}
\begin{abstract}
    A description of how text, math, images, and interactive content are
    rendered.
\end{abstract}
\maketitle

Rendering in Ximera can be broken down into three parts:
\begin{description}
    \item[Rendering text] is handled using the \LaTeX\ package TeX4ht.
    \item[Rendering math] is handled using MathJax via TeX4ht.
    \item[Rendering images] is done by directly showing the image (in the case
        of a PNG or JPG) or converting to SVG and displaying the SVG.
\end{description}


There are issues with variables like ``textwidth'' versus
``pagewidth'' used in \LaTeX. For online rendering, these variables are all
basically made to be the width of the browser window (more or less---there are
some very technical details here). This can make it difficult to horizontally
align things with any subtlety. Moreover, the author should not trust \LaTeX\ for vertical spacing on the web, as our page is very long and it is unclear what various units mean.





\end{document}
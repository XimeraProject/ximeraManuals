\documentclass{ximera}

\title{Getting and giving help}
\begin{document}
\begin{abstract}
    Sometimes the manual just isn't enough!
\end{abstract}
\maketitle

If you encounter any issues or have questions while using Ximera, there are
multiple ways to get assistance:

\paragraph{Email the Ximera developers} who are happy to help! We can assist
you with any technical or content-related questions you may have. You can reach
us directly via email:
\begin{center}
    \url{ximera@math.osu.edu}
\end{center}

When contacting us, please include: The \textbf{URL to the GitHub repository}
as this will allow us to directly see and interact with your code and the
\textbf{URL of a deployed page} if applicable.

\paragraph{Join our Discord server} by going to:
\begin{center}
    \url{https://go.osu.edu/author-discord}
\end{center}
There you can ask questions and talk directly to developers.

\paragraph{Submit an issue on our GitHub} and we'll find it!
If you have identified a bug, want to request a feature, you can make the
request here:
\begin{center}
    \url{https://github.com/XimeraProject/ximeraLatex/issues}
\end{center}
If the Ximera Manual requires clarification, please submit an issue on our
GitHub repository:

\begin{center}
    \url{https://github.com/XimeraProject/ximeraManuals/issues}
\end{center}

When submitting an issue, follow these steps:
\begin{enumerate}
    \item Log in to your GitHub account. If you don't have one, you can create
          an account for free at \href{https://github.com}{GitHub}.
    \item Navigate to:
          \url{https://github.com/XimeraProject/ximeraManuals/issues}
    \item Click the ``New Issue'' button.
    \item Provide a descriptive title and a detailed explanation of the issue
          or suggestion.
    \item Attach any relevant files, screenshots, or links to help the team
          understand your report.
\end{enumerate}

\paragraph{Community contributions} are essential to Ximera's sustainability.
Ximera is an open-source project, and we welcome contributions from the
community. If you have suggestions for improving the platform or this manual,
feel free to fork the repository and submit a pull request on GitHub.

If you would like to contribute to the Ximera document class, we suggest the following workflow:
\begin{enumerate}
\item Make a \textbf{Template} of a testing repository like: \link{https://github.com/XimeraProject/ximeraFirstSteps}
\item \textbf{Clone your template}, or work on your template in a GitHub Codespace.
\item \textbf{Fork} XimeraLatex.
\item \textbf{Clone your fork} of Ximera \LaTeX\ \textbf{into your template}, and switch to the \verb!development! branch (this should be up-to-date with the \verb!master! branch).
\item \textbf{In your template}, rename the folder \verb!ximeraLatex! to \verb!.ximera_local!. Now your template will use this version of Ximera \LaTeX.
\item Make your edits by either directly editing a \textbf{preamble file} (that overrides the current \verb!ximeraLatex!) or within \verb!ximera.cls! found within \verb!.ximera_local! directly.
\item Once your changes work, make your changes in the \verb!*.dtx! files. If you developed in \verb!ximera.cls! directly, change its name to \verb!ximeradev.cls! so that it is not overwritten in the next step.
\item Run \verb!make! and a new \verb!ximera.cls!, \verb!xourse.cls!, \verb!ximera.4ht! and \verb!xourse.4ht! will be generated, overwriting any existing files in \verb!.ximera_local!.
\item Push changes to your fork of \verb!ximeraLatex!. It is best to make small commits, and give detailed descriptions.
\item When ready, \textbf{submit a pull-request} from your fork.
\end{enumerate}
Once your changes have been accepted via a pull-request, Ximera developers will update the branch master from the branch development.


Thank you for using Ximera. We are committed to providing high-quality tools
for creating and sharing interactive mathematics content. Your feedback and
questions are invaluable to us!

\end{document}
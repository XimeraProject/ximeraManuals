\documentclass{ximera}

%% Where to find images
\graphicspath{  %% When looking for images,
{./}            %% look first at your level,
{./basics/}     %% then in this folder,
}

% might be overwritten in printstyle, to get a proper back cover for this course
\providecommand{\backCover}{}

\title{Set up a GitHub repository}

\author{Bart Snapp}

\begin{document}
\begin{abstract}
  How to set up your Ximera files in GitHub.
\end{abstract}
\maketitle

\paragraph{All Ximera files must be hosted in a git repository} To deploy, you
will also need	additional
files found in \verb!ximeraFirstSteps!. You have two choices when creating new
content:
\begin{itemize}
  \item If you are \textbf{starting fresh}, fork the \texttt{ximeraFirstSteps}
        repository.
  \item If you are starting with an \textbf{existing a git repository}, that is
        not currently deploying Ximera content, move files
        from \texttt{ximeraFirstSteps} to this repository.
\end{itemize}
We'll address each of these methods in-turn below.
% See the \link[git manual]{http://git-scm.com} for more information about the
% \verb!clone! and \verb!fork! commands.

\section{Starting fresh and forking \texttt{ximeraFirstSteps}}

Forking a repository is well-documented on
\link[GitHub]{https://docs.github.com/en/pull-requests/collaborating-with-pull-requests/working-with-forks/fork-a-repo}.
Basically, you login to GitHub, return to this page, and at the top right there
will be an option to \textbf{fork} this repository. Fork the repo. Accept all
defaults, though you might want to change the name of the repository at this
point. When done, it will take you to
your copy of this repository on GitHub. It will be located someplace like:
\begin{center}
  \texttt{https://github.com/YOUR-GIT-USER-NAME/your-new-repo-name}
\end{center}
Once the repository is forked, clone the forked repository (the one in your
user-space) onto your computer. \textbf{If you are using Windows, be sure to
  clone through WSL.}
After the repository is on your computer, delete all files \textbf{except:}
\begin{itemize}
  \item \texttt{.gitignore}
  \item \texttt{DOTximeraserve}
  \item \texttt{.vscode}
  \item \texttt{scripts}
  \item \texttt{.git}
  \item \texttt{README.md}
  \item \texttt{NOT-THE-LICENSE.md}
\end{itemize}
You must keep the files above.	Typically we advise the CC BY-NC-SA 4.0
License; however, you do you.
Commit your changes to GitHub and view your
repository online via a web browser to ensure that the files were deleted.

\section{Starting from an existing GitHub repository}

Starting from an existing GitHub repository, you will need to \textbf{copy}
\begin{itemize}
  \item \texttt{.gitignore}
  \item \texttt{.vscode}
  \item \texttt{scripts}
  \item \texttt{NOT-THE-LICENSE.md} (only if you don't already have a license)
\end{itemize}
into your repository. If you already have a \verb|.gitignore| file, we suggest
you replace yours with ours. Commit your changes to GitHub and view your
repository online via a web browser to ensure that the files were added.

Typically we advise the CC BY-NC-SA 4.0 License; however, you do you.
Finally, after the \verb!.gitignore! is pushed to the repository, copy
\verb!.ximerserve! to your repository.


\end{document}

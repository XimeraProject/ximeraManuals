\documentclass{ximera}

\title{Tools to author Ximera content}
\author{Bart Snapp}

\begin{document}
\begin{abstract}
    Tools for working on Ximera documents.
\end{abstract}
\maketitle

Authors own their content in Ximera and work on their own machine.
Nevertheless, with the assistance of Visual Studio Code and Docker, we are able
to provide a common deployment environment for MacOS, Windows, and Linux.

%\xmsection{Docker and Visual Studio Code}

\paragraph{Docker} is a development utility that allows us to set up a computer
within your
computer. You must remember to start Docker before you deploy. You can just let
it run in the background. You can check the status of Docker by running
\verb!docker ps! in a terminal session.

\paragraph{Visual Studio Code}
is a popular text-editor. It is has strong \LaTeX, GitHub,
and Docker integration. It provides a UNIX-like command line interface for
Windows machines via WSL. A typical workflow would be to:
\begin{enumerate}
    \item Open Docker and minimize the window.
    \item Open Visual Studio Code, do File $\to$ Open Folder, and select the
          folder of your git repository.
    \item To open files, do \verb!Ctrl-p! and start typing file names. Any
          file
          committed to your git repository will be found, and files in
          \verb!.gitignore!, will not be shown.
    \item To run a special command (like search and replace) do
          \verb!Ctrl-Shift-p!, and search for the command.
    \item To toggle a UNIX-like terminal, use \verb!Ctrl-~!.
\end{enumerate}


\end{document}
\documentclass{ximera}

\author{Wim Obbels and Bart Snapp}

\title{First steps in Ximera}

\begin{document}
\pdfOnly{\onecolumn\begin{multicols}{2}}
        \begin{abstract}
            Try out Ximera!
        \end{abstract}
        \maketitle

        To use Ximera, you must have a \link[GitHub]{https://github.com}
        account.
        GitHub is a web platform where developers can store, share, and
        manage
        their code. It uses \verb!git!, popular software for version control,
        to help
        teams work together simultaneously
        without overwriting each other's changes. GitHub has	issue tracking,
        pull requests for proposing changes, and other project management
        tools. It's
        like a shared folder
        for coding, designed to help teams work smarter and track progress.
        Go to \url{https://github.com} and either sign-up or log-in. Note, you
        must
        know
        your \textbf{username} and \textbf{password}, so store them in a safe
        place; like in a safe, or under your bed. After you have a GitHub
        account, log-in and go to:
        \begin{center}
            \url{https://github.com/ximeraProject/ximeraFirstSteps}
            %% 230% on my horiz screen
        \end{center}

        You will see something like this:
        \pdfOnly{\end{multicols}}
\begin{image}
    \includegraphics[width=.7\textwidth]{xfsTemplate}
\end{image}
\newpage

\begin{image}
    \includegraphics[width=.7\textwidth]{xfsCreate}
\end{image}
\pdfOnly{\begin{multicols}{2}}
        Click on the green ``Use this template'' button and select ``Create a
        new repository.'' Give it a fun repository name, and push the button
        ``Create repository.''
        At this point you have your own personal copy of our repository
        \verb!ximeraFirstSteps!.
        In fact, after you create it, GitHub will take you to it. This copy can
        always
        be found at
        \begin{center}
            \verb!https://github.com/YOUR-GIT-USER-NAME/YOUR-REPO-NAME!
        \end{center}
        \newpage
        \pdfOnly{\end{multicols}}
\begin{image}
    \includegraphics[width=.7\textwidth]{xfsCodespace}
\end{image}
\pdfOnly{\begin{multicols}{2}}
        Once there, click the green ``Code'' button, select
        the ``Codespaces'' tab, and click ``Create codespace on main.''
        A GitHub codespace is a remote computer set up specifically for coding.
        We have our codespace preconfigured with all the tools, libraries, and
        software
        you need for a Ximera
        project. With a codespace, you can instantly start working
        without worrying about setting up software on your local machine.
        Moreover,
        others can go to your GitHub page, start their own codespace, and try
        out your
        code.
        \textbf{It will take around 5 minutes for your codespace to be created
            and you
            must wait until it is complete.}
        \pdfOnly{\end{multicols}}

\newpage

\begin{image}
    \includegraphics[width=.7\textwidth]{xfsVScode}
\end{image}
\pdfOnly{\begin{multicols}{2}}
        Once the codespace is created, you will see something like what we have
        above.
        This is Visual Studio Code (VS Code) running within your browser. This
        is a powerful text-editor with many extensions. One could use it write Ximera
        content.  On the far left, you see a
        vertical list of icons. Currently, ``EXPLORER'' is selected, it looks
        like
        ``pages of paper.'' Moving right, we see the files in our GitHub Repo.
        At the bottom right-hand corner of the screen you will see a button
        that says
        ``SERVE.'' This button was added by the files in \verb!.vscode/!.
        Press the ``SERVE'' button to compile Ximera content to HTML and
        JavaScript.
        This will take a few minutes. When the compilation is finished, note the line that says:
        ``PROBLEMS,'' ``OUTPUT,'' ``DEBUG CONSOLE,'' ``TERMINAL,'' ``PORTS.''
        \pdfOnly{\end{multicols}}

\newpage
 
\begin{image}
    \includegraphics[width=.7\textwidth]{xfsPorts}
\end{image}
\pdfOnly{\begin{multicols}{2}}
You want
to click on ``PORTS.'' The ``PORTS'' tab may be hidden within ``$\cdots$.''
After you click on ``PORTS,'' select 2080 and click on the globe, and a webpage will open. Your
content will be under the link ``Content.'' You should be able to see the
content in your browser.


You may delete your codespace (you can simply restart it) and others can come to your GitHub repository, start a codespace, and check out and compile your code. 
This is especially useful if a user runs into difficulty, as a Ximera developer can examine a users exact setup, and help resolve any issues. 


Demo versions of this repository are published as:
\begin{itemize}
    \item \link[a preview with the newest Ximera layout]{https://set.kuleuven.be/voorkennis/firststeps24/aFirstXourseVariant/aFirstFolder/aFirstActivityVariant}
    \item \link[a version on the current production server]{https://ximera.osu.edu/firststeps24/aFirstXourse/aFirstFolder/aFirstActivity}
\end{itemize}

\pdfOnly{\end{multicols}}
\pdfOnly{\twocolumn}
\end{document}
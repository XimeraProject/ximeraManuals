\documentclass{ximera}

\title{About Ximera}
\author{Bart Snapp}

\begin{document}
\begin{abstract}
    What is Ximera? What is it supposed to do? Who is it for?
\end{abstract}
\maketitle

Ximera, pronounced ``chimera,'' (\textbf{X}imera: \textbf{I}nteractive,
\textbf{M}athematics, \textbf{E}ducation,
\textbf{R}esources, for \textbf{A}ll) is an open-source platform that provides
tools for
authoring and publishing (PDF and Online), open-source, interactive educational
content, such as textbooks, assessments, and online courses.

\paragraph{Authors}  write and store their content on their own
machines and GitHub repositories.
Authors own their content and decide how to license their content. From a
single source written in \LaTeX, Ximera generates various output: PDF
worksheets, PDF textbooks, and	PDF solution manuals, and so on. Of most interest,
Ximera can also create online interactive activities:
\begin{image}
    \begin{tikzpicture}
        \node at (-1.8,.2) {\resizebox{1cm}{!}{../xmPictures/document.tex}};   %%NOT SHOWING UP ONLINE!
        \node at (-1.8,.2) {\small PDF};
        
        \node at (-2,0) {\resizebox{1cm}{!}{../xmPictures/document.tex}};    %%NOT SHOWING UP ONLINE!
        \node at (-2,0) {\small PDF};
        
        \node at (-2.2,-.2){\resizebox{1cm}{!}{../xmPictures/document.tex}};    %%NOT SHOWING UP ONLINE!
        \node at (-2.2,-.2) {\small PDF};
        \draw[->] (-.6,0) -- (-1.3,0);
        \draw[->] (-.6,.3) -- (-1.3,.3);
        \draw[->] (-.6,-.3) -- (-1.3,-.3);

        \node at (0,0) {\resizebox{1cm}{!}{../xmPictures/document.tex}};    %%NOT SHOWING UP ONLINE!
        \node at (0,0) {\LaTeX};
        \draw[->] (.6,0) -- (1.4,0);

        \node at (2,0) {\resizebox{1cm}{!}{\begin{tikzpicture}[rounded corners=.5pt]
    \draw (0,0) rectangle (1.2,1.35);
    \draw (.4,0) -- (.4,1.35);
    \draw (.8,0) -- (.8,1.35);

    \draw (0,0)+(.85,1.1) rectangle ([shift={(.85, 1.1)}] .3, .2);
    \draw (0,0)+(.85,.85) rectangle ([shift={(.85, .85)}] .3, .2);
    \draw (0,0)+(.85,.1) rectangle ([shift={(.85, .1)}] .3, .2);

    \draw (1.2,0) -- (1.2,1.35) -- (1.5,1.65) -- (1.5,.3) -- cycle;

    \draw (0,1.35) -- (.3,1.65) -- (1.5,1.65) -- (1.2,1.35) -- cycle;
    \draw (.4,1.35) -- (.7,1.65);
    \draw (.8,1.35) -- (1.1,1.65);    
\end{tikzpicture}}};
        \draw[->] (2.6,0) -- (3.4,0);
        \node at (4,0) {\resizebox{1cm}{!}{../xmPictures/computer.tex}};     %%% FIX WITH PIC TECHNOLOGY
        
        \node[anchor=north] at (-2,-.8) {Various};
        \node[anchor=north] at (-2,-1.2) {PDFs};

        \node[anchor=north] at (0,-.8) {Single};
        \node[anchor=north] at (0,-1.2) {Source};

        \node[anchor=north] at (2,-.8) {Deploy};
        \node[anchor=north] at (2,-1.2) {Server};

        \node[anchor=north] at (4,-.8) {Students};
        \node[anchor=north] at (4,-1.2) {Engage};
    \end{tikzpicture}
\end{image}
The source code used to produce PDFs can also create interactive online
activities when deployed to a Ximera server. Students access this content via a
URL or an assignment in their LMS.


\paragraph{Students} interact with the \textit{content} produced within
Ximera, hence their experience is highly dependent on the \textit{quality} of this
content. Research shows that students find Ximera materials to be more readable
than traditional course materials and perform equivalently to those using
proprietary textbooks and online homework systems. While students typically
encounter Ximera through their courses, many discover it via web-search and
use the platform as independent learners. In 2023, Ximera had over one million
unique visitors. Since Ximera materials are free, they are accessible to
anyone, regardless of enrollment in official courses.




\paragraph{Get involved} by contributing as an instructor, author,
or developer. To get started with Ximera, visit our
\textit{First Steps in Ximera} GitHub repository:
\begin{center}
    % \qrcode{https://go.osu.edu/xfs}\\
    \url{https://github.com/ximeraProject/ximeraFirstSteps}
\end{center}

\paragraph{Funding for the Ximera Project} is provided by
a U.S.\ Department of Education Open Textbooks Pilot Program grant in the
amount of \$2,125,000, from 2024--2026, with no external funding.
In the past, the Ximera Project has
also received support from NSF Grant DUE-1245433, the Shuttleworth
Foundation, the Ohio State University
Department of Mathematics, and the Affordable Learning Exchange at Ohio State.

As a token of our appreciation, \textbf{consider applying for a Ximera
    Flash-Grant Stipend:}
\begin{center}
    % \qrcode{https://go.osu.edu/ximera-flash-grant}\\
    \url{https://go.osu.edu/ximera-flash-grant}
\end{center}
Thank you for your interest in Ximera. We encourage you to contact the
team with any questions you may have.

\paragraph{The authors} listed on the cover are the current Ximera lead
developers. In reality, this document has many authors as it is part of an
evolution of Ximera documentation. Rodney Austin, Marcus Bishop, Oscar Levin, Matt Thomas, and Hans
Parshall authored parts of either the document class or original
documentation.
\end{document}

\documentclass{ximera}
\title{Rendering content}

\author{Jason Nowell \and Bart Snapp}
\begin{document}
\begin{abstract}
    A description of how text, math, images, and interactive content are
    rendered.
\end{abstract}
\maketitle

Rendering in Ximera can be broken down into three parts:
\begin{description}
    \item[Rendering text] is handled using the \LaTeX\ package TeX4ht.
    \item[Rendering math] is handled using MathJax via TeX4ht.
    \item[Rendering images] is done by directly showing the image (in the case
        of a PNG or JPG) or converting to SVG and displaying the SVG.
\end{description}

\section{Basics of rendering}

There are issues with variables like ``textwidth'' versus
``pagewidth'' used in \LaTeX. For online rendering, these variables are all
basically made to be the width of the browser window (more or less---there are
some very technical details here). This can make it difficult to horizontally
align things with any subtlety.






\section{Accessibility}
MathJax has extensive accessibility features built in, which means
Ximera benefits from the developers keeping this up-to-date and conforming to
industry standard. In essence, you don't need to worry about accessibility
features for rendered math content - with the exception of graphs.

Due to how graphs are rendered, they currently don't have any
accessibility features (e.g. alt-text) if you provide them via TikZ or other
LaTeX means. You can input them as image files instead, however those also lack
any accessibility support. This is something to keep in mind, as you may need
to provide textual description explicitly for things like screen readers to
provide more accessibility for graphs and/or images.


\end{document}
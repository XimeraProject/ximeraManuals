\documentclass{ximera}

%% Where to find images
\graphicspath{  %% When looking for images,
{./}            %% look first at your level,
{./basics/}     %% then in this folder,
}

% might be overwritten in printstyle, to get a proper back cover for this course
\providecommand{\backCover}{}

\title{Set up a GitHub repository}

\author{Bart Snapp}

\begin{document}
\begin{abstract}
  How to set up your Ximera files in GitHub.
\end{abstract}
\maketitle

\paragraph{All Ximera files must be hosted in a git repository} To deploy, you
will also need	additional
files found in \verb!ximeraNewProject!. We suggest you make your own personal copy of  \verb!ximeraNewProject!, and build your new Ximera projects directly there.
This process is called \textit{forking a GitHub repository}.



\section{Forking \texttt{ximeraNewProject}}

Forking a repository is well-documented on
\link[GitHub]{https://docs.github.com/en/pull-requests/collaborating-with-pull-requests/working-with-forks/fork-a-repo}.
Basically, you login to GitHub, return to this page, and at the top right there
will be an option to \textbf{fork} this repository. Fork the repo. Accept all
defaults, though you want to \textbf{change the name of the repository} at this
point, we'll use ``YOUR-REPO-NAME'' for this in the discussion below. When done, it will take you to
your copy of this repository on GitHub. It will be located someplace like:
\begin{center}
  \texttt{https://github.com/YOUR-GIT-USER-NAME/YOUR-REPO-NAME}
\end{center}
Once the repository is forked, clone the forked repository (the one in your
user-space) onto your computer. \textbf{If you are using Windows, be sure to
  clone through WSL.} To do this, open Visual Studio Code, and hit \verb!Ctrl-~! to open a terminal window (select WSL if you are on Windows), and run
\begin{verbatim}
git clone LINE FROM GIT
\end{verbatim}
After the repository is on your computer, you should change the names of a few files. Change the name of \verb!newCourse.tex!, to be a reasonable, web-safe name, say \verb!GOOD-NAME.tex!. 
To do this run
\begin{verbatim}
git mv newCourse.tex GOOD-NAME.tex
\end{verbatim}
You'll also want to set the license to be something other than the Public Domain. We've included the CC BY-NC-SA 4.0
License as a suggested license. You can either edit this to be another license or move it on to the other license with
\begin{verbatim}
git mv SUGGESTED-LICENSE.md LICENSE.md
\end{verbatim}
Then do
\begin{verbatim}
git add -u 
git commit -m "set course name and license"
git push
\end{verbatim}
Now 

\end{document}

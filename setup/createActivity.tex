\documentclass{ximera}

%% Where to find images
\graphicspath{  %% When looking for images,
{./}            %% look first at your level,
{./basics/}     %% then in this folder,
}

% might be overwritten in printstyle, to get a proper back cover for this course
\providecommand{\backCover}{}

\title{Create an activity}

\author{Bart Snapp}

\begin{document}
\begin{abstract}
    How to set up an activity
\end{abstract}
\maketitle

It is easy to write a basic Ximera document with clean \LaTeX. The structure is
as follows:

\begin{verbatim}
\documentclass{ximera}
%%% Where to find images
\graphicspath{  %% When looking for images,
{./}            %% look first at your level,
{./basics/}     %% then in this folder,
}

% might be overwritten in printstyle, to get a proper back cover for this course
\providecommand{\backCover}{} %% You need to give a correct path to the preamble.
\author{Your name}
\title{Your title} %% Not needed for drill exercises.
%\license{What ever you choose} %% Only necessary if you need different licenses for different activities.
\outcomes{list your outcomes,another outcome}  %% Only if you intend to do something with the outcomes
\begin{document}
\begin{abstract}                             %% Not needed for drill exercises
A one-sentence description of the activity.  %% Not needed for drill exercises
\end{abstract}                               %% Not needed for drill exercises
\maketitle

%%
%% Content goes here
%%
\end{document}
\end{verbatim}

\paragraph{Use standard \LaTeX} environments and commands like:
\begin{description}
    \item[enumerate]
    \item[itemize]
    \item[description]
    \item[\verb!align*!]
    \item[Math Mode]
    \item[Macros] made of basic \LaTeX\ commands are fine.
\end{description}
\textbf{No manual formatting.}
\end{document}

\documentclass{ximera}

%% Where to look for inputs
\makeatletter     %% make "@" a letter-character
\def\input@path{  %% When looking for files,
{./}              %% look first at your level
{./coverArt/}     %% then in this folder,
{./introduction/} %% then in this folder,
}
\makeatother      %% make "@" an other-character

%% Where to find images
\graphicspath{                      %% When looking for images, 
{./}                                %% look first at your level,
{./setup/}                          %% then in this folder,
{./graphicsVideosAndInteractives/}  %% then in this folder,
}



% Custom Commands


%% These may change and should be checked! 
\newcommand{\docURL}{\url{https://github.com/ximeraProject/ximeraFirstSteps/ximeraUserManual}}
\newcommand{\testURL}{\url{https://github.com/ximeraProject/tests}}
\newcommand{\experimentalURL}{\url{https://github.com/ximeraProject/experimental}}
\newcommand{\xfsURL}{https://go.osu.edu/xfs}
\newcommand{\xfgURL}{https://go.osu.edu/ximera-flash-grant}

\newcommand{\event}[1]{\def\theevent{#1}}
\newcommand{\theevent}{}


\newif\ifcolor %% for color cover
\colorfalse
\colorlet{bkgndcr}{white}
\colorlet{txtcr}{black}
\colorlet{otherbkgndcr}{white}
\colorlet{othertxtcr}{black}


\usepackage{qrcode} %% For QR Codes
\usepackage[normalem]{ulem} % for strikeout
\usepackage{multicol} % multicols -- PDF only

\makeatletter
%% Make Front style
\newcommand\frontstyle{%
  \def\activitystyle{activity-chapter}
  \def\maketitle{%
                {\flushleft\small\sffamily\bfseries\@pretitle\par\vspace{-1.5em}}%
                {\flushleft\LARGE\sffamily\bfseries\@title \par }%3
                {\vskip .6em\noindent\textit\theabstract\setcounter{problem}{0}\setcounter{section}{0}}%
                \par\vspace{2em}    
                \phantomsection\addcontentsline{toc}{section}{\textbf{\@title}}%
                \setcounter{titlenumber}{0}
}}

\renewcommand\chapterstyle{%
  \def\activitystyle{activity-chapter}
  \normalsize
  %\onecolumn
  \def\maketitle{%
    \addtocounter{titlenumber}{1}%
                    {\flushleft\small\sffamily\bfseries\@pretitle\par\vspace{-1.5em}}%
                    {\flushleft\LARGE\sffamily\bfseries\thetitlenumber\hspace{1em}\@title \par }%
                    {\vskip .6em\noindent\textit\theabstract\setcounter{problem}{0}\setcounter{section}{0}}%
                    \par\vspace{2em}
                    \phantomsection\addcontentsline{toc}{section}{\textbf{\thetitlenumber\hspace{1em}\@title}}%
}}

%% Redefine section and subsection
\renewcommand\section{\@startsection {section}{1}{\z@}%
                                   {-3.5ex \@plus -1ex \@minus -.2ex}%
                                   {2.3ex \@plus.2ex}%
                                   {\boldmath\normalfont\large\bfseries\sffamily}}
\renewcommand\subsection{\@startsection{subsection}{2}{\z@}%
                                     {-3.25ex \@plus -1ex \@minus -.2ex}%
                                     {1.5ex \@plus .2ex}%
                                     {\boldmath\normalfont\large\bfseries\sffamily}}


\renewcommand\paragraph{\@startsection{paragraph}{4}{\z@}%
                                    {3.25ex \@plus1ex \@minus.2ex}%
                                    {-1em}%
                                    {\boldmath\normalfont\normalsize\bfseries\sffamily}}
\makeatother

\title{Create an activity}

\author{Bart Snapp}

\begin{document}
\begin{abstract}
    How to set up an activity
\end{abstract}
\maketitle

It is easy to write a basic Ximera document with clean \LaTeX. The structure is
as follows:

\begin{verbatim}
\documentclass{ximera}
%%% Where to look for inputs
\makeatletter     %% make "@" a letter-character
\def\input@path{  %% When looking for files,
{./}              %% look first at your level
{./coverArt/}     %% then in this folder,
{./introduction/} %% then in this folder,
}
\makeatother      %% make "@" an other-character

%% Where to find images
\graphicspath{                      %% When looking for images, 
{./}                                %% look first at your level,
{./setup/}                          %% then in this folder,
{./graphicsVideosAndInteractives/}  %% then in this folder,
}



% Custom Commands


%% These may change and should be checked! 
\newcommand{\docURL}{\url{https://github.com/ximeraProject/ximeraFirstSteps/ximeraUserManual}}
\newcommand{\testURL}{\url{https://github.com/ximeraProject/tests}}
\newcommand{\experimentalURL}{\url{https://github.com/ximeraProject/experimental}}
\newcommand{\xfsURL}{https://go.osu.edu/xfs}
\newcommand{\xfgURL}{https://go.osu.edu/ximera-flash-grant}

\newcommand{\event}[1]{\def\theevent{#1}}
\newcommand{\theevent}{}


\newif\ifcolor %% for color cover
\colorfalse
\colorlet{bkgndcr}{white}
\colorlet{txtcr}{black}
\colorlet{otherbkgndcr}{white}
\colorlet{othertxtcr}{black}


\usepackage{qrcode} %% For QR Codes
\usepackage[normalem]{ulem} % for strikeout
\usepackage{multicol} % multicols -- PDF only

\makeatletter
%% Make Front style
\newcommand\frontstyle{%
  \def\activitystyle{activity-chapter}
  \def\maketitle{%
                {\flushleft\small\sffamily\bfseries\@pretitle\par\vspace{-1.5em}}%
                {\flushleft\LARGE\sffamily\bfseries\@title \par }%3
                {\vskip .6em\noindent\textit\theabstract\setcounter{problem}{0}\setcounter{section}{0}}%
                \par\vspace{2em}    
                \phantomsection\addcontentsline{toc}{section}{\textbf{\@title}}%
                \setcounter{titlenumber}{0}
}}

\renewcommand\chapterstyle{%
  \def\activitystyle{activity-chapter}
  \normalsize
  %\onecolumn
  \def\maketitle{%
    \addtocounter{titlenumber}{1}%
                    {\flushleft\small\sffamily\bfseries\@pretitle\par\vspace{-1.5em}}%
                    {\flushleft\LARGE\sffamily\bfseries\thetitlenumber\hspace{1em}\@title \par }%
                    {\vskip .6em\noindent\textit\theabstract\setcounter{problem}{0}\setcounter{section}{0}}%
                    \par\vspace{2em}
                    \phantomsection\addcontentsline{toc}{section}{\textbf{\thetitlenumber\hspace{1em}\@title}}%
}}

%% Redefine section and subsection
\renewcommand\section{\@startsection {section}{1}{\z@}%
                                   {-3.5ex \@plus -1ex \@minus -.2ex}%
                                   {2.3ex \@plus.2ex}%
                                   {\boldmath\normalfont\large\bfseries\sffamily}}
\renewcommand\subsection{\@startsection{subsection}{2}{\z@}%
                                     {-3.25ex \@plus -1ex \@minus -.2ex}%
                                     {1.5ex \@plus .2ex}%
                                     {\boldmath\normalfont\large\bfseries\sffamily}}


\renewcommand\paragraph{\@startsection{paragraph}{4}{\z@}%
                                    {3.25ex \@plus1ex \@minus.2ex}%
                                    {-1em}%
                                    {\boldmath\normalfont\normalsize\bfseries\sffamily}}
\makeatother %% You need to give a correct path to the preamble.
\author{Your name}
\title{Your title} %% Not needed for drill exercises.
%\license{What ever you choose} %% Only necessary if you need different licenses for different activities.
\outcomes{list your outcomes,another outcome}  %% Only if you intend to do something with the outcomes
\begin{document}
\begin{abstract}                             %% Not needed for drill exercises
A one-sentence description of the activity.  %% Not needed for drill exercises
\end{abstract}                               %% Not needed for drill exercises
\maketitle

%%
%% Content goes here
%%
\end{document}
\end{verbatim}

\paragraph{Use standard \LaTeX} environments and commands like:
\begin{description}
    \item[enumerate]
    \item[itemize]
    \item[description]
    \item[\verb!align*!]
    \item[Math Mode]
    \item[Macros] made of basic \LaTeX\ commands are fine.
\end{description}
\textbf{No manual formatting.}
\end{document}

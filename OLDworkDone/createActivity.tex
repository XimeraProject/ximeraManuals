\documentclass{ximera}

\title{Create an activity}

\author{Bart Snapp}

\begin{document}
\begin{abstract}
    How to set up an activity.
\end{abstract}
\maketitle

It is easy to write a basic Ximera document with clean \LaTeX. The structure is
as follows:
\begin{verbatim}
\documentclass{ximera}
\begin{document}
%%
%% Content goes here
%%
\end{document}
\end{verbatim}

A Ximera activity may be a single problem, a list of exercises, a review of a
concept, a section of a book---you are only limited by your imagination.
Examples of individual Ximera activities can be found here:
\begin{itemize}
    \item blah
\end{itemize}



\paragraph{Use standard \LaTeX} environments and commands like:
\verb!enumerate!,
\verb!itemize!,
\verb!description!,
\verb!align*!, \verb!$...$!, \verb!\[...\]!, and so on.
Macros made of basic \LaTeX\ commands are fine. If you store them in \verb!xmPreamble.tex!
\textbf{No manual formatting.}

\paragraph{Special Ximera documents and folders} are \textbf{automatically}
loaded by all files in the repository (assuming they are no more than three folders deep).
\begin{description}
    \item[\texttt{xmPreamble.tex}]
    \item[\texttt{xmPictures/}]
\end{description}




\end{document}

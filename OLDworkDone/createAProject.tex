\documentclass{ximera}

\title{Create a Ximera project}

\author{Bart Snapp}

\begin{document}
\begin{abstract}
    How to create a new Ximera project.
\end{abstract}
\maketitle

\paragraph{All Ximera files must be hosted in a git repository.}
As an author, you have some choices as to how you proceed.

\section{GitHub setup}

\paragraph{Using a template} is easy.
We have two repositories potential authors can use as a template:
\begin{itemize}
    \item \url{https://github/ximeraProject/ximeraFirstSteps} for a template
          with multiple files that you can edit.
    \item \url{https://github/ximeraProject/ximeraNewProject} for a more
          minimal template.
\end{itemize}
If you go to either of these repositories when you are logged in to GitHub, you
will have the option to make a new repository using the green ``Use this
template'' button.

\paragraph{To deploy an existing repository from a codespace,} just copy the files
you will need from either \verb!ximeraFirstSteps! or \verb!ximeraNewProject!
\begin{itemize}
\item \texttt{.devcontainer/} (a folder you need if you want to deploy from a GitHub codespace)
\item \texttt{xmScripts/} (this is a folder)
\item \texttt{.vscode/} (this is a folder)
\item \texttt{.xmKeyFile} (if you want to deploy from a GitHub codespace)
\item \texttt{.gitignore} 
\end{itemize}
into your repository. If you already have a \verb|.gitignore| file, we suggest
you \textbf{replace yours with ours.} Commit your changes to GitHub and view your
repository online via a web browser to ensure that the files were added.


If you want to deploy via codespace, add


Finally, after the \verb!.gitignore! is pushed to the repository.





\end{document}
\documentclass{ximera}

\title{Authoring tools}
\author{Bart Snapp \and Wim Obbels}

\begin{document}
\begin{abstract}
      Tools for working on Ximera documents.
\end{abstract}
\maketitle

Authors own their content in Ximera and can work on their own machine. This has
been a fundamental aspect of Ximera since the beginning. 
%However, this can make
%development difficult, since we have at least three platforms to support:
%MacOS, Windows, and Linux.
To write a Ximera document authors can use their preferred editor, and compile to PDF 
with a local TeXLive or MikTeX installation that has the Ximera package, 
as available from \href{https://ctan.org/pkg/ximera}{CTAN}.

However, one needs some extra software to generate and deploy the online version. 
This is most conveniently provided through Visual Studio Code and Docker.
Moreover, the version control system \verb!git! is needed.
And, when using Docker a local TeX installation is no longer necessary.

%\xmsection{Docker and Visual Studio Code}

\section{Docker}

Docker is a development utility allow you to easily run self-contained applications on your
computer. You must install and start Docker before you can deploy. You can just let
it run in the background. You can check the status of Docker by running
\verb!docker ps! in a terminal session, provided by Visual Studio Code.

\section{Visual Studio Code}

Visual Studio Code
is a popular text-editor. It is has strong \LaTeX, git,
and Docker integration. It provides a UNIX-like command line interface on
Windows machines via Microsoft's WSL (Windows Subsystem for Linux).

\paragraph{A typical workflow} would be to:
\begin{enumerate}
      \item Open Docker and minimize the window.
      \item Open Visual Studio Code, do File $\to$ Open Folder, and select the
            folder of your git repository.
      \item To open files, do \verb!Ctrl-p! and start typing file names. Any
            file
            committed to your git repository will be found, and files in
            \verb!.gitignore!, will not be shown.
      \item To run a special command (like search and replace) do
            \verb!Ctrl-Shift-p!, and search for the command.
      \item To toggle a UNIX-like terminal, use \verb!Ctrl-~!.
\end{enumerate}

\paragraph{Visual Studio Code can be enhanced} with a number of extensions. 
Upon opening the ximeraFirstSteps repo for the first time, you'll be asked to install some 
extensions (that are needed for the recommended Ximera workflow)
\begin{itemize}
      \item LaTeX Workshop by James Yu (or LaTeX by Mathematic Inc for a more minimal version)  
      \item Task Buttons by spencerwmiles
      \item Git
\end{itemize}

% When you use GitHub, you are writing code in public. You don't want randos to
% be able to change your code, and they cannot.  But that same protection might make it 
% difficult for you to develop. So we need to set up a bit of security, called
% SSH, so that you can authenticate but others cannot.
% Once we are finished with that, we will give you some basic \texttt{git}
% commands.

% \paragraph{Setting up SSH} GitHub requires some security. One way to do this is
% to add an SSH key. To check to see if this is already done, open Visual Studio
% Code, open a (WSL) terminal with \verb!Ctrl-~! and enter
% \begin{verbatim}
% git remote -v
% \end{verbatim}
% if it responds with something like: If it responds with
% \verb!git@github.com:STUFF! then you
% are good and can skip this. If not, and it responds with
% \verb!https://github.com/STUFF! the you
% need to change to SSH, read on!

% First we must create the key. There are instructions on \link[GitHub's pages]%
% {https://docs.github.com/en/authentication/connecting-to-github-with-ssh/generating-a-new-ssh-key-and-adding-it-to-the-ssh-agent}
% Once the key is created, go to your personal GitHub page, like
% \verb!https://github.com/bartsnapp! and:
% \begin{enumerate}
%       \item  Click on \textit{Settings}, it lives in the circular picture in
%             upper right hand corner.
%       \item Click on \textit{SSH and GPG keys}, it's on the left.
%       \item Click the green button \textit{New SSH key}
%       \item Name it \textit{My Ximera Key}.
%       \item Paste in your Key.
% \end{enumerate}
% Now when you clone a repository, you will use the SSH option.  To convert a
% repository from the HTTPS option to SSH,
% \begin{verbatim}
% git remote set-url origin git@github.com:STUFF
% \end{verbatim}
% You should be ready to go at this point.

\section{\texttt{git}}
\texttt{git} is program for version control. It allows many authors
to work on the same document without fear
of overwriting others work. We will discuss \texttt{git} in more detail later.
It keeps a \textit{complete history} of your work, and can be reverted to a
previous state. 
Generating a PDF from a Ximera document is possible without using git, but in order to
generate HTML and to deploy the documents your project has to be stored in a git repository.
(Is is not absolutely necessary to publish this repo on GitHub or similar)

\paragraph{Use \texttt{git} in Visual Studio Code} to add files, commit, push and pull.
Alternatively, you can also use the universal and powerful git command line interface from a 
terminal window, which we'll briefly introduce next.

\paragraph{If you make a new file,} you need to \textit{add} it to the
repository. Suppose the name of your new file was MY-NEW-FILE.tex. You should
type:

\begin{verbatim}
git add MY-NEW-FILE.tex
git commit -m "I've add MY-NEW-FILE.tex"
git push
\end{verbatim}
The \verb!add! includes the file in the files that \verb!git! tracks. The
\verb!commit! says, ``hey I made a change that I want to keep track of.'' The
\verb!push! says ``I want to share this with others.''

\paragraph{If you update a file,}
type use \verb!git add -u! and it will add all updated files.
\begin{verbatim}
git add -u
git commit -m "I updated some files"
git push
\end{verbatim}
You can combine these commands into one with \verb!&&!:
\begin{verbatim}
git add -u && git commit -m "this is my change" && git push
\end{verbatim}

\paragraph{When you commit} you must type a message, describing the commit. You
don't have to leave a \textit{good} message, but you must say something.
If you just type \verb!git commit! you may end up in \verb!vi!, the terminal
will look like:
\begin{verbatim}
~
~
~
~
~
~
\end{verbatim}
In this case, take a deep breath and
\begin{center}
      press \verb!Esc! \quad then press \verb!:!\quad then press \verb!q! \quad
      then
      press \verb#!# \quad then press \verb!Enter!
\end{center}
Then run
\begin{verbatim}
git config --global core.editor "nano"
\end{verbatim}
This will place you in a more user friendly editor in the future.

\paragraph{If you are collaborating,} you should always start your work off
with
\begin{verbatim}
git pull
\end{verbatim}
typed in terminal. This will retrieve any updates your collaborators have made.
Then you do your work, make sure your files compile, then do
\begin{verbatim}
git add -u && git commit -m "this is my change" && git push
\end{verbatim}

\paragraph{Conflicts} sometimes occur even when using \verb!git!. When this
happens you'll need to \textit{resolve} the conflict. To do this, open the file
where the conflict is.
\begin{verbatim}
git status
\end{verbatim}
should tell you this. Once you find the file, open it, and search for ``HEAD'' and you'll find something like
\begin{verbatim}
<<<<<<< HEAD
Some stuff that some person wrote.
=======
Some other stuff that someone else wrote
>>>>>>> 3d13c5bba8dc2c12b7783e4f85315b1773165fd6
\end{verbatim}
In this case just edit the lines starting with !<<<<<<<! and ending with the line \verb!>>>>>>>! and make it look how it should.
Then do
\begin{verbatim}
git add -u
git commit -m "I updated some files"
git push
\end{verbatim}
and the conflict should be resolved.

% \paragraph{Adding images and PDFs with \texttt{git}}

% We typically add \verb!*.jpg!, \verb!*.png! and \verb!*.pdf! to the
% \verb!.gitignore! file. However this makes it a little tricky to add graphics.
% In this case you \textit{force} the add with \verb!-f!:
% \begin{verbatim}
% git add -f SOME-GRAPHICS.jpg
% \end{verbatim}

\end{document}

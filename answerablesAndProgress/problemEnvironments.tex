\documentclass{ximera}

\title{Problem environments and nesting}
\author{Bart Snapp}

\begin{document}
\begin{abstract}
\end{abstract}
\maketitle

% \section{Expanding problem types}

One thing you might want to have are problems that unfold as
students work. We have special environments for this.

While \textbf{any} environment can contain the command \verb|\answer|,
there are four special environments: \verb|question|, \verb|exercise|,
\verb|problem|, \verb|exploration|. If these environments are nested within
each other, They \textit{hide} the inside environment. For example

\begin{problem}
Start with $F_0 = 1$ and $F_1=1$. Define
\[
  F_{n+1} = F_n + F_{n-1} \qquad\text{for $n\ge$ 1}
\]
Find $F_2$.
\[
  F_2 = \answer{2}
\]
\begin{problem}
Find $F_3$
\[
  F_3 = \answer{3}
\]
\end{problem}
\end{problem}
With this said, this technique should be used with care and perhaps even
sparingly. When problems unfold, students don't have any idea when the
\textit{pain} will end. If we consider the unfolding problems above, but simply
listed in order:
\begin{verbatim}
\begin{problem}
Start with $F_0 = 1$ and $F_1=1$. Define
\[
  F_{n+1} = F_n + F_{n-1} \qquad\text{for $n\ge$ 1}
\]
Find $F_2$.
\[
  F_2 = \answer{2}
\]
\end{problem}
\begin{problem}
Find $F_3$
\[
  F_3 = \answer{3}
\]
\end{problem}
\end{verbatim}
little is lost pedagogically and a student has a clear end in sight.



\end{document}
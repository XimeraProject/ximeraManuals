\documentclass{ximera}

\title{Select correct answers with \textbackslash choice}
\author{Bart Snapp}

\begin{document}
\begin{abstract}
  Let students select correct answers.
\end{abstract}
\maketitle

%\section{Answers with \texttt{\textbackslash choice}}

There are three similar answerables that all use the command \verb!\choice!. 

\begin{warning}
  All answerables must be \textbf{inside of a theorem-like environment}.
\end{warning}


In each case that uses \verb!\choice!, the order of the choices presented to the student is the order the
author types in the code. The author marks correct answers with the option
\verb!correct! and leaves incorrect answers without options.
\begin{verbatim}
\choice[correct]{SOME-CORRECT-ANSWER}
\choice{SOME-INCORRECT-ANSWER}
\end{verbatim}
Now we will discuss each specific environment that uses \verb!\choice!.
\paragraph{Multiple choice}
questions, including True/False, are intended for students to select one 
correct answer.
\begin{verbatim}
\begin{question}
Which of the following functions has a graph which is a parabola?
\begin{multipleChoice}
  \choice[correct]{$y=x^2+3x-3$}
  \choice{$y = \frac{1}{x+2}$}
  \choice{$y=3x+1$}
\end{multipleChoice}
\end{question}
\end{verbatim}
If more than one choice is labeled correct with \verb!multipleChoice!, any
correct answer will result in completion of this answerable.
\begin{verbatim}
\begin{problem}
Select a prime number:
\begin{multipleChoice}
  \choice{1}
  \choice[correct]{2}
  \choice[correct]{3}
  \choice{4}
  \choice[correct]{5}
\end{multipleChoice}
\end{problem}
\end{verbatim}
With \verb!multipleChoice! the student is only able to select one answer before
submitting. This could be useful for student surveys where every choice is
marked as correct.

\paragraph{Select all} problems allow the student to select any and all answers
before
submitting.
\begin{verbatim}
\begin{problem}
Select all prime numbers:
\begin{selectAll}
  \choice{1}
  \choice[correct]{2}
  \choice[correct]{3}
  \choice{4}
  \choice[correct]{5}
\end{selectAll}
\end{problem}
\end{verbatim}

Select all problems can be very challenging for students. Authors can quickly
make
questions that are quite difficult without realizing it.

\paragraph{Word choice}
problems were designed for inline \textit{words}; however, at this
point we
do support math in the choices. We give an example of \verb!\wordChoice! in
action below:
\begin{verbatim}
\begin{exercise}
 Consider the planes defined by the equations below.
\begin{align*}
  P_1:  \quad 4 &= 2x-y+3z  \\
  P_2:  \quad 5 &= 4x-2y+6z \\ 
  P_3:  \quad 7 &= 5x+2y+z
\end{align*}
Describe the relationships between the planes 
$P_1$, $P_2$, and $P_3$ in terms of ``parallel,'' 
``orthogonal,'' or ``neither.''
\begin{enumerate}
  \item The planes $P_1$ and $P_2$ are \wordChoice{
    \choice[correct]{parallel}
    \choice{orthogonal}
    \choice{neither parallel nor orthogonal}
    }
  \item The planes $P_1$ and $P_3$ are \wordChoice{
    \choice{parallel}
    \choice{orthogonal}
    \choice[correct]
    {neither parallel nor orthogonal}}.
  \item The planes $P_2$ and $P_3$ are \wordChoice{
    \choice{parallel}
    \choice{orthogonal}
    \choice[correct]{neither parallel nor orthogonal}
    }
\end{enumerate}
\end{exercise}
\end{verbatim}
It is difficult to have a PDF version of \verb!\wordChoice! (unless you use the
documentclass option \texttt{wordchoicegiven} so authors should
take this into consideration.


\end{document}
